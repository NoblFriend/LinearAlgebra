\documentclass[../main.tex]{subfiles}
\begin{document}
    \begin{note}
        При циклическом процессе первое начало термодинамики можно интерпретировать как закон сохранения энергии:
        \begin{equation}
            \oint \delta Q - \oint \delta A = 0.
        \end{equation} 
        Такой подход свойственен теоретической физике.
    \end{note}
\subsection{Про неквазистатический процесс}
    Рассмотрим газ под поршнем, атмосферой пренебрежём, будем считать, что газа $\nu = 1$~моль. Масса поршня $M$. Запишем давление и уравнение состояния:
    \begin{gather}
        P_1 = \frac{Mg}{S}, \\
        MgH_1 = RT_1,
    \end{gather}
    где $H_1$~--- высота положения поршня. Теперь \emph{резко} положим на него что-то заметной (существенной, иначе~--- сравнимой с $M$) массы $m$. Сначала начнет происходить что-то неописуемое (хоть и адиабатически), но однажды всё придёт в равновесие. Запишем закон сохранения энергии сразу после удара и после установления равновесия:
    \begin{equation}
        \label{eq:lec2:problem}
        (M+m)gH_1+\frac{3}{2}RT_1=(M+m)gH_2+\frac{3}{2}RT_2.
    \end{equation}

    \begin{note}
        Немного преобразуем равенство \eqref{eq:lec2:problem}:
        \begin{align}
            \frac{(m+M)g}{S} (H_1-H_2)S &= \frac{3}{2} R (T_2 - T_1)\\
            \underbrace{P_{\text{out}} \Delta V_{\text{out}}}_{\text{Работа внешних сил}} &= \underbrace{\frac{3}{2} R (T_2 - T_1).}_{_{\text{Изменение внутренней энергии}}}
        \end{align}
    \end{note}
    \begin{note}
        Немного преобразуем равенство \eqref{eq:lec2:problem} и уравнение состояния:
        \begin{align}
            (M+m)\underbrace{\frac{RT_1}{M}}_{gH_1}+\frac{3}{2}RT_1 &= (M+m)\underbrace{\frac{RT_2}{M+m}}_{gH_1}++\frac{3}{2}RT_2, \\
            \left(\frac{M+m}{M} \right)RT_1 + \frac{3}{2}RT_1 &= \frac{5}{2}RT_2, \\
            RT_1\left(\frac{3}{2} + \frac{M+m}{M}\right)&=\frac{5}{2}RT_2,
        \end{align}
        с учётом $P(m) = mg/S \sim m$ заменим отношение масс:

    \end{note}
        
\end{document}
\documentclass[../main.tex]{subfiles}
\begin{document}

    \section{Термодинамические системы}
    \begin{definition}
        \emph{Термодинамическая система (ТС)}~--- содержит большое количество частиц, окруженное термостатом (вообще всем остальным). 
    \end{definition}
    \begin{note}
        Вселенная не является ТС, так как нет ограничений (то есть, у неё нет термостата).
    \end{note}
    
    \begin{definition}
        \emph{Изолированная система}~--- система, которая вообще никак не взаимодействует с окружающей средой.
    \end{definition}
    
    \subsection{Минус-первое начало термодинамики}
    Это утверждение иногда называют нулевым началом термодинамики, но мы делать так не будем, ведь есть другое нулевое начало термодинамики. Так же часто его называют \emph{основным} или \emph{общим}. В общем, знатный рецидивист.

    \begin{proposition}[Минус-первое начало термодинамики]
        Любая изолированная система придёт в равновесие, её можно охарактеризовать макроскопическими параметрами и она не может сама покинуть состояние равновесия.
    \end{proposition}
        
    \begin{note}
        Время релаксации зависит от размеров системы.
    \end{note}
    \begin{note}
        У некоторых объектов существует метастабильное состояние.
    \end{note}

    \subsection{Классификация систем}
        \begin{definition}
            Над \emph{адиабатической (или теплоизолированной) ТС} мы можем только совершать работу (не можем передавать тепло, изменять количество частиц).
        \end{definition}    
        \begin{definition}
            Система, к которой мы можем только подводить тепло, называется \emph{калорической} (по Колдунову).
        \end{definition}
        \begin{definition}
            Над \emph{закрытой системой} можно совершать работу и подводить к ней тепло, но нельзя менять состав частиц.
        \end{definition}
        \begin{definition}
            У \emph{открытой системы} можно изменять только количество частиц. 
        \end{definition}
        \begin{note}
            Комбинация открытой системы с другими даёт нам окончательную классификацию.
        \end{note}

        \begin{definition}
            Термодинамическая система с двумя каналами называется \emph{простой}.
        \end{definition}
       

        \section{Температура}
        \begin{definition}
            \emph{Экстенсивная величина} пропорциональна количеству частиц.
        \end{definition}
        \begin{note}
            Экстенсивные величины аддитивны.
        \end{note}
        \begin{example}
            Объём, количество частиц.
        \end{example}

    Величины можно разделить на экстенсивные~--- аддитивные $(N, V)$, пропорциональные количеству частиц, можно мерить по своим эталонам,
    и интенсивные $(T, p)$, для измерения которых нужно привязывать сторонние параметры.

    Методы измерения температуры:
    
        1) По расширению\\
        2) По сопротивлению\\
        3) Другие\\
        4) Газовый термометр $T \leftrightarrow P$

    Нулевое начало термодинамики

    Если два тела находятся в термодинамическом равновесии с третьим, то тогда они в термодинамическом равновесии между собой.
    
    Уравнение состояния (Термическое уравнение)

    Замечание. В Термодинамике оно всегда существует в независимости от того как его получать.

    *Вывод формулы*

    Математические отступления:

        1) разложение на частные производные

        2) Не важен порядок взятия частных производных у хороших функций

        3) Свойство циклической перестановки

    Вывод (3) для P, V, T

    Термические коэффициенты

    \begin{eqnarray}
        \alpha  =   \frac{1}{V} \cdot \extpartder{V}{T}{P} \\
        \beta   =   \frac{1}{P} \cdot \extpartder{P}{T}{V} \\
        \chi    = - \frac{1}{V} \cdot \extpartder{V}{P}{T}
    \end{eqnarray}

    \begin{equation}
        \frac{\chi \beta}{\alpha} = \frac{1}{P}
    \end{equation}

    Первое начало термодинамики 

    Принцип Майера: Работа пропорциональна теплоте?

    Квазистатический процесс

    Неквазистатический процесс\\
    $\delta Q$, $\delta A$,~--- функции процесса\\
    $dU$~--- функция состояния\\

    $$ \delta Q + \delta A = dU$$



\end{document}
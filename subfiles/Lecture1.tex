\documentclass[../main.tex]{subfiles}
\begin{document}

    "Не хотел я это начинать. Я попробовал разобраться, как понял расскажу, не судите строго."\\
    "Заботал как вы за одну ночь перед экзаменом"
    По поводу литературы,
    чтобы решать задачи советую Кириченко\\
    Сивухин\\
    Пригожин\\
    Ко всей термодинамике, которую вы читаете, относитесь критично, сколько человек
    столько и мнений.

    Термодинамическая система~--- содержит большое количество частиц, окруженное термостатом. 
    Он может влиять на систему влияя на V, P, T, ЭМ поля. Вселенная не окружена термостатом.

    Изолированная система~--- никак не взаимодействует с окружением.

    Основное или Общее или -1 ("Знатный рецидивист") начало термодинамики:\\
        Любая Изолированная система придёт в равновесие\\
        Можно характеризовать макроскопическими параметрами\\
        Не может сама покинуть состояние равновесия\\
    Замечание 1. Время релаксации зависит от размеров системы\\
    Замечание 2. У некоторых объектов существует метастабильное состояние

    Классификация систем

        1) Адиабатическая теплоизолированная ТС

        2) ТС к которой подводят только тепло, "Калорическая" по Колдунову

        3) Закрытая система, к которой можно подводить тепло и совершать работу, но
        нельзя менять состав частиц

        4) Открытая ТС - у которой можно менять количество частиц.

        ТС с двумя каналами называется простой

    Температура
    
    Величины можно разделить на экстенсивные~--- аддитивные $(N, V)$, пропорциональные количеству частиц, можно мерить по своим эталонам,
    и интенсивные $(T, p)$, для измерения которых нужно привязывать сторонние параметры.

    Методы измерения температуры:
    
        1) По расширению\\
        2) По сопротивлению\\
        3) Другие\\
        4) Газовый термометр $T \leftrightarrow P$

    Нулевое начало термодинамики

    Если два тела находятся в термодинамическом равновесии с третьим, то тогда они в термодинамическом равновесии между собой.
    
    Уравнение состояния (Термическое уравнение)

    Замечание. В Термодинамике оно всегда существует в независимости от того как его получать.

    *Вывод формулы*

    Математические отступления:

        1) разложение на частные производные

        2) Не важен порядок взятия частных производных у хороших функций

        3) Свойство циклической перестановки

    Вывод (3) для P, V, T

    Термические коэффициенты

    $$\displaystyle \alpha = \frac{1}{V} \cdot (\frac{\partial V}{\partial T})_{p} \\$$
    $$\displaystyle \beta = \frac{1}{P} \cdot (\frac{\partial P}{\partial T})_{v} \\$$
    $$\displaystyle \chi = - \frac{1}{V} \cdot (\frac{\partial V}{\partial P})_{T} \\$$

    $$\displaystyle \frac{\chi \beta}{\alpha} = \frac{1}{P}$$

    Первое начало термодинамики 

    Принцип Майера: Работа пропорциональна теплоте?

    Квазистатический процесс

    Неквазистатический процесс\\
    $\delta Q$, $\delta A$,~--- функции процесса\\
    $dU$~--- функция состояния\\

    $$ \delta Q + \delta A = dU$$



\end{document}
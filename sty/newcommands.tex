\renewcommand{\phi}{\ensuremath{\varphi}}
\renewcommand{\kappa}{\ensuremath{\varkappa}}
\renewcommand{\le}{\ensuremath{\leqslant}}
\renewcommand{\leq}{\ensuremath{\leqslant}}
\renewcommand{\ge}{\ensuremath{\geqslant}}
\renewcommand{\geq}{\ensuremath{\geqslant}}
\renewcommand{\emptyset}{\ensuremath{\varnothing}}

\DeclareMathOperator{\sgn}{sgn}
\DeclareMathOperator{\Arg}{Arg}
\DeclareMathOperator{\Ln}{Ln}
\DeclareMathOperator{\re}{Re}
\DeclareMathOperator{\im}{Im}
\DeclareMathOperator{\Arsh}{Arsh}
\DeclareMathOperator{\Int}{int}
\DeclareMathOperator{\cl}{cl}

\newcommand{\N}{\mathbb{N}}
\newcommand{\Z}{\mathbb{Z}}
\newcommand{\Q}{\mathbb{Q}}
\newcommand{\R}{\mathbb{R}}
\newcommand{\Cm}{\mathbb{C}}
\newcommand{\F}{\mathbb{F}}
\newcommand{\id}{\mathrm{id}}
\newcommand{\goth}{\mathfrak}
\newcommand{\mc}{\mathring}
\newcommand{\eps}{\varepsilon}
\newcommand{\veps}{\epsilon}
\newcommand{\vdelta}{\partial}
\newcommand{\Tau}{\mathcal{T}}
\newcommand{\dvec}[1]{\Delta\vec{#1}}
\newcommand{\wt}[1]{\widetilde{#1}}
\newcommand{\liml}{\lim\limits}
\newcommand{\suml}{\sum\limits}
\newcommand{\prodl}{\prod\limits}
\newcommand{\such}{\ \big|\ }
\newcommand{\range}[1]{\{1, \ldots, #1\}}

\newcommand{\System}[1]{
	\left\{\begin{aligned}#1\end{aligned}\right.
}
\newcommand{\Root}[2]{
	\left\{\!\sqrt[#1]{#2}\right\}
}
\newcommand{\Matrix}[1]{
	\left(\begin{aligned}#1\end{aligned}\right)
}
\newcommand{\Det}[1]{
	\left|\begin{aligned}#1\end{aligned}\right|
}
\newcommand{\trbr}[1]{
	\left\langle#1\right\rangle
}

\newcommand{\partder}[2]{\ensuremath{\dfrac{\partial #1}{\partial #2}}}
\newcommand{\extpartder}[3]{\ensuremath{\left(\partder{#1}{#2}\right)_{#3}}}

\renewcommand\labelitemi{$\triangleright$}

\theoremstyle{plain}
\newtheorem{theorem}{Теорема}[section]
\newtheorem{lemma}{Лемма}[section]
\newtheorem{proposition}{Утверждение}[section]
\newtheorem*{exercise}{Упражнение}

\theoremstyle{definition}
\newtheorem{definition}{Определение}[section]
\newtheorem*{adefinition}{Определение(не материал лектора)}
\newtheorem*{corollary}{Следствие}
\newtheorem*{addition}{Дополнение}
\newtheorem*{note}{Замечание}
\newtheorem*{anote}{Замечание автора}
\newtheorem*{reminder}{Напоминание}
\newtheorem*{example}{Пример}
\newtheorem*{examples}{Примеры}

\theoremstyle{remark}
\newtheorem*{solution}{Решение}

%%% Перенос знаков в формулах (по Львовскому)
\newcommand*{\hm}[1]{#1\nobreak\discretionary{}{\hbox{$\mathsurround=0pt #1$}}{}}

\newcommand{\segment}{\subsubsection*}

%%% Нумерация уравнений
\makeatletter
\def\eqref{\@ifstar\@eqref\@@eqref}
\def\@eqref#1{\textup{\tagform@{\ref*{#1}}}}
\def\@@eqref#1{\textup{\tagform@{\ref{#1}}}}
\makeatother                      % \eqref* без гиперссылки
\numberwithin{equation}{section}  % Нумерация вида (номер_секции).(номер_уравнения)
\mathtoolsset{showonlyrefs=true} % Номера только у формул с \eqref{} в тексте.
